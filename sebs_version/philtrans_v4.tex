\documentclass [11pt,fleqn]{article}

\usepackage{amssymb}
%\usepackage{epsf,psfig,graphicx}
\usepackage{epsf,graphicx}
\usepackage{psfrag}

\usepackage{color}
\topmargin     -0.60in  % (adjusted for printer bias) 
\headheight      .00in  % (no headers) 
\headsep         .50in  % (top margin + headers + skip) 
\textheight     9.50in  % (instructions: 9 1/8'' min, 9 7/16'' max) 
\textwidth      6.00in  % 2*3.33 + .33 = 6.99 
\oddsidemargin  0.3125in  % (subtracted 1inch bias) 
\evensidemargin 0.3125in 
%\renewcommand{\baselinestretch}{1.5} 
\parindent .0in 
\parskip 10pt 

\font \bigtenrm=cmmi10 scaled\magstep2 
\def \dt {\delta\tau} 
\def \ve {\varepsilon} 
\def \ch {{\cal H}} 
\def \del {\partial} 
\def \be {\begin{equation}} 
\def \ee {\end{equation}} 
\def \beq {\begin{eqnarray}} 
\def \eeq {\end{eqnarray}} 
\def \tv {\tilde v} 
\def \veren {\varepsilon^{\rm ren}_f} 
\def \vef {\varepsilon^0_f} 
\def \su {\uparrow} 
\def \sd {\downarrow} 
\def \CR {\nonumber\\} 
\def \hfb {\hfill\break} 
\def \tb {\bar{t} } 
\def \kb {\bar{k} } 
\def \tbB {\bar{t}_B } 
%\def \ul{#1} {$\underline{#1 }$}

\begin{document} 
\hspace{1cm}{\bf New Title:} Observation of correlated x-ray scattering at atomic resolution \hfb
 
{\bf Authors:} Derek Mendez, Thomas J. Lane, Jongmin Sung,  Cl\`ement Levard, Herschel Watkins, Aina Cohen, Michael Soltis, Shirley Chisholm, James Spudich, Vijay Pande,  Daniel Ratner, Sebastian Doniach

{\bf Abstract}


\section{Introduction}

In a pioneering paper, Kam [1] showed that correlated x-ray scattering (CXS) from an ensemble of randomly oriented particles could in principle reveal information about the internal structure of the particles beyond usual small and wide angle solution scattering measurements. The extraction of such information in the absence of an ordered system ( e.g. a crystal) can be beneficial in biological studies, as many biological processes are inherently disordered (e.g. proteins in solution ).

In order to gauge the feasibility of Kam's method at atomic length scales, and to assess the difficulties associated with such a sensitive measurement, we conducted experiments measuring CXS from silver nanoparticle (NP) solutions at wide angles. Crucially, each measurement was conducted on an ensemble of NPs oriented randomly in three dimensions, extending previous work done in two dimensions  [2], at small angles in 3-dimensions [3-4], and on single particles in 3-dimensions [5-6].  

By successfully measuring CXS signal from an ensemble of silver NPs, we have demonstrated the effectiveness of Kam's method given the current advances in x-ray technology. This experiment will serve as a benchmark for future experiments involving much weaker scatterers, such as proteins. It is our hope that the refinement of our our analysis techniques on a well known sample such as silver NPs will help facilitate the extension of CXS to studies of biomolecules in solution.

\section{Theory}

Let $S( \vec{q},\omega)$ represent the structure factor of an oriented particle in solution, i.e.

\[
S(\vec{q},\omega) = | \int \rho (\hat{R}(\omega)\cdot \vec{r} ) e^ { i \vec{q} \cdot \vec{r} } d\vec{r} | ^{2}
\]

where $\rho $ is the electron density, $\omega$ is a triple of Euler angles, and $\hat{R}(\omega)$ is a 3-dimensional rotation operator.


Kam showed that the correlation function

\be
C(\vec{q}_1, \vec{q}_2) \equiv \int S( \vec{q}_{1},\omega ) S( \vec{q}_{2},\omega ) \, d \omega
\ee

may be extracted from a CXS measurement where one repeatedly records snapshots of a solution of $N$ identical particles, with each snapshot representing a unique ensemble of particle orientations. Kam argued (neglecting inter-particle scattering interference ) that averaging correlations of intensity fluctuations on each snapshot would result in the construction of (1) , i.e.

\be
\langle \delta n_{k}(\vec{q}_1) \, \delta n_{k}(\vec{q}_2) \rangle_{k} \Rightarrow C(\vec{q}_1, \vec{q}_2) 
\ee

where $n_{m}(\vec{q})$ is the total photons scattered from all particles into momentum transfer vector $\vec{q}$ on snapshot $m$ and $\delta n_{k}(\vec{q}) = n_{k}(\vec{q}) - \langle n_{k}(\vec{q}) \rangle_{k}$.
 
For the case of silver nanoparticles we can represent them by the simplest model consisting of a face-centered-cubic lattice cutoff by a spherical boundary. The scattering can then be represented by reciprocal lattice vectors cutting the Ewald sphere and giving rise to Bragg peaks broadened owing to the the finite size of the nanoparticles. Hence, each snapshot records a series of ``powder rings" corresponding to the different vectors of the reciprocal lattice, and the different particle orientations. We denote the magnitude of the strongest Bragg ring by $|\vec{q}_{111}|$ which is equal to the space diagonal of the reciprocal lattice body-centered-cubic primitive cell, $|\vec{q}_{111}|=\sqrt{3}a/2=2.66  $\AA$ ^{-1}$. Each NP has 8 reciprocal lattice points at this magnitude. Using the Scherrer formula relating the width of a Bragg peak to the particle size (our NP distribution was peaked at 20 nm) , along with geometrical arguments discussed below, we find that $\approx 8.3\%$ of silver NP orientations will contribute to the first powder ring. Call this set of orientations $\Omega_{111,1}$. The sub-population of NP orientations in $\Omega_{111,1}$ that simultaneously subtend a second Bragg peak on the Ewald sphere at $|\vec{q}_{111}|$ will give rise to an auto-correlation signal along the $|\vec{q}_{111}|$ powder ring. Our best estimates suggest that $\approx 0.17\%$ of orientations in $\Omega_{111,1}$  satisfy this criterion, or roughly $1.1\%$ of all NP orientations. Call this set of orientations $\Omega_{111,2}$. We arrive at these numbers by tracing out volumes of rotation of the reciprocal lattice points and determining which fractions intersect the Ewald sphere. To determine $\Omega_{111,1}$ we rotate the reciprocal lattice points over all possible orientations. To determine $\Omega_{111,2}$ we rotate the reciprocal lattice about one of the 8 vectors with magnitude $|\vec{q}_{111}|$, hence fixing 1 lattice point on the Ewald sphere and rotating until another intersects. These arguments are simple in nature, and deserve to be properly verified. We simulated these volumes of intersection and found $\Omega_{111,1}$ and $\Omega_{111,2}$ to represent $8.8\%$ and $1.3\%$ of all NP orientations respectively, in good agreement with our geometrical arguments [figure ?].


\section{Methods}
A snapshot in (2) is supposed to represent an ensemble of particles frozen at an instant in time, however exposure times are finite. Therefore, the sample was cooled to 100 Kelvin using a nitrogen cryo jet to ensure that the particles remained immobilized during each exposure. This also ensured that  any heating due to the x-ray beam was minimal. We had an estimated $10^{9}$  20 nm NPs per snapshot, but we observed significant numbers of NPs that were up to 50 nm in size.  The NPs were held in colloidal suspension with glycerol-based antifreeze used to prevent the formation of solvent crystals at the low temperature. By
monitoring the intensity at constant scattering angle one can check for sample
damage and diffusion.

To house the solutions we used kapton capillaries with a 500 and 600 $\mu$m inner
and outer diameter respectively. Kapton scatters into relatively lower angles as
does glycerol, hence we did not anticipate corrupting our silver nanoparticle signal with scattering from the kapton or glycerol.  The experiment was conducted at the micro-crystallography beamline (12-2) at
SSRL. Samples were prepared a day early and stored in a liquid nitrogen bath.
Samples were loaded and oriented in the X-ray beam using the Stanford Automated Mounting System (SAM ) , controllable from the experimental hutch.
Using a liquid nitrogen-cooled double crystal monochromator we tuned the beam
energy to 17 keV. The beam was focused down to about $20 \times 50 \mu$m$^2$ using Rh
coated Kirkpatrick-Baez mirrors.
Snapshots were recorded on a Dectris Pilatus 6M pixel detector. Our goal was to record as many snapshots as possible, each one representing a different ensemble of particle orientations frozen in time. The sample holder was equipped to automatically 
rotate the capillary about its longitudinal axis. The capillary was oriented such that
its longitudinal axis was perpendicular to the beam (Figure needed) and then it
was rotated through a 150 degree angle. The photon counts were read out and
reset every 0.7 second and every 0.3 degrees or rotation, giving us 500 shots per
150 degree rotational scan. This was deemed an optimal timing to simultaneously
maximize signal and minimize damage and heating. The beam was shining on
the capillary for the duration of the scan, hence the center of the capillary was
constantly heated. However, this effect was negligible given the path length of
the beam through the sample (500 $\mu$m). Between scans we moved the capillary
longitudinally, so as to always probe different regions of the sample, and hence
different ensembles of particle orientations.

The intensity fluctuation correlations in (2) were constructed by auto- and cross- correlating the powder rings on each snapshot. For example, assuming an area detector perpendicular to the x-ray beam defined by the coordinates $(q^{\,\perp},\phi)$, the auto-correlation function of the $q_{111}$ powder ring
\beq
\left \langle \delta n_{k}(\vec{q}_{111,1}) \, \delta n_{k}(\vec{q}_{111,2}) \right \rangle_{k}  &=& \left \langle \delta n_{k} ( q_{111},\phi_{1}, \theta_{111} ) \,\, \delta n_{k} (q_{111},\phi_{2}, \theta_{111}  ) \right \rangle_{k} \nonumber \\
&\Rightarrow& \left \langle \int_{0}^{2\pi} \delta n_{k} ( q_{111}^{\,\perp},\phi ) \,\, \delta n_{k} (q_{111}^{\,\perp},\phi + \Delta\, \phi )\, d\phi  \right \rangle_{k}
\eeq
should exhibit CXS peaks at values of $\Delta\, \phi = \phi_{1} - \phi_{2}$ corresponding to the geometrical separation of two NP Bragg reflections at $q_{111}$.

Data from about 15,496 shots was collected and analyzed. A bicubic interpolation algorithm was used to convert the cartesian pixel lattice to polar coordinates for calculation of equations like (3).

\section{Results}

In order to eliminate any source of systematic noise in the intensity measurement
we applied a binary  filter to the data: intensities greater than a chosen threshold
were set to unity, the rest were set to zero ( Figure 1 ). The resulting auto correlations averaged to display a pair of peaks corresponding to the double Bragg scattering discussed above rising above the background due to the UDS events (fig 2)
%\begin{figure}[h]
%\begin{center}
%\includegraphics[height=7cm]{./fig3.pdf}
%\end{center}
%{\bf Figure 1:} Plot of the average of  $>$ 15,000 auto correlators computed from the first Bragg ring (dark line).  Shading around the line  is the standard error, vertical lines denote the correlator peaks from the double Bragg theory. The lower curve is the result of a  simulation of scattering from 1000 randomly oriented 20 nm nanoparticles.
%\end{figure}

The signal/noise in these measurements results both from the intrinsic UDS noise and from the Poisson statistics of photon scattering. To gauge the relative magnitudes of these sources of noise we took advantage of the photon counting capability of the Pilatus detector. 

The total elastic scattering integrated over all angles is $n_{\rm scattered photons}\simeq\Phi\ N \sigma_{\rm nanoparticle}$ (up to a constant of $\cal O$(1)). Here $N$ is the number of particles in the beam $\Phi$ is the x-ray fluence and $\sigma_{\rm NP}$ is the coherent part of the x-ray scattering cross section for a nanoparticle (calculated from the atomic cross section taken from tables.) Integrating the photon counts over all 5 Bragg rings which are accessed by the 17keV x-rays, we get a total scattered photon count of order $10^9$ photons per shot. The x-ray fluence was $2\ 10^{12}$ photons into a focal spot of $20 \times 50 \mu$m$^2$ for a 0.5 second exposure. From the tabulated coherent cross section, and putting in 260,000 silver atoms/NP we estimate a scattering rate of 1.6 photons per NP per shot. We conclude that there were of order 6.3\ $10^8$ NPs in the beam.

From the estimates of the fraction of NPs giving CDS events given above, we conclude that the observed correlated scattering results from $\approx 0.1\%$ of the total scattering, giving an order of magnitude estimate of  $10^6$ CDS events per shot, thus showing that Poisson noise is not limiting the determination of the correlator profile in our simple case. Since our experiments show that averaging over a few thousand shots is adequate to separate the CDS events from the UDS background we conclude that the main impediment to accurate measurements of the correlated scattering comes from systemic errors resulting from  anisotropy artifacts induced by  the detector system, which we have been able to partially overcome by the nonlinear filtering to a binary signal. 

\section{Discussion}
As originally shown by Kam, the correlator for scattering from an arbitrary molecule averaged over random orientations, $C(\vec{q}_1,\vec{q}_2,\psi)$, with $\psi$ the angle between the scattering vectors, may be expanded in a series of Legendre polynomials  in $\psi$ whose coefficients may be directly calculated from the scattering structure factors of the molecule $F(\vec {q})=\sum_i f_i(q) \exp(i\vec {q}\cdot\vec {r}_i)$ where $\vec {r}_i$ are the atomic positions and $f_i(q)$ the atomic x-ray form factors. Hence accurate measurement of $C$ in the 3-dimensional $\{\vec{q}_1,\vec{q}_2,\psi\}$ space can lead to constraints which can be placed on the atomic positions, thus giving a route to iterative refinement of a given model (ref Brunger). 

Our results show that it is possible to obtain atomic scale information on the internal structure, for the very simple example  of a silver nanoparticle, for a bulk sample containing of order $10^8$ identical but randomly oriented particles. These results suggest that it should be feasible to obtain more detailed atomic scale constraints on models  of more complex biomolecules by measuring scattering using x-ray pulses from xFELs (refs Hajdu, Chapman - Spence). Such measurements have the potential to scatter many more photons/molecule, yielding more detailed $q_1,q_2,\psi$ information on the correlators. 

\section{References}

[1] Kam Z 1977 Determination of macromolecular structure in solution by spatial correlation of scattering fluctuations. Macromolecules 10 927-34

[2] Saldin D K, Poon H C, Bogan M J, Marchesini S, Shapiro D A, Kirian R A, Weierstall U and Spence J C H 2011 New light on disordered ensembles: ab initio structure determination of one particle from scattering fluctuations of many copies, Phys. Rev. Lett. 106 115501

[3] Kam Z, Koch M H J, and Bordas J 1981 Fluctuation x-ray-scattering from biological particles in frozen solution by using synchrotron radiation. Proc. Natl. Acad. Sci. USA 78 3559-62

[4] Wochner P, Gutt C, Autenrieth T, Demmer T, Bugaev V, Ortiz A D, Duri A, Zontone F, Grubel G and Dosch H 2009 X-ray cross correlation analysis uncovers hidden local symmetries in disordered matter. Proc. Natl Acad. Sci. USA 106 11511-4

[5] Kam Z and Gafni I 1985 3-dimensional reconstruction of the shape of human wart virus using spatial correlations. Ultramicroscopy 17 251�62

[6] Starodub D \textit{et al.} 2010 Single-particle structure determination by correlations of snapshot x-ray diffraction patterns. Nature Communications 3. (DOI: 10.1038/ncomms2288)

\end{document}






