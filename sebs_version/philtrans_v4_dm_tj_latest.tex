\documentclass [11pt,fleqn]{article}

\usepackage{amssymb}
%\usepackage{epsf,psfig,graphicx}
\usepackage{epsf,graphicx}
\usepackage{psfrag}

\usepackage{color}
\topmargin     -0.60in  % (adjusted for printer bias) 
\headheight      .00in  % (no headers) 
\headsep         .50in  % (top margin + headers + skip) 
\textheight     9.50in  % (instructions: 9 1/8'' min, 9 7/16'' max) 
\textwidth      6.00in  % 2*3.33 + .33 = 6.99 
\oddsidemargin  0.3125in  % (subtracted 1inch bias) 
\evensidemargin 0.3125in 
%\renewcommand{\baselinestretch}{1.5} 
\parindent .0in 
\parskip 10pt 

\font \bigtenrm=cmmi10 scaled\magstep2 
\def \dt {\delta\tau} 
\def \ve {\varepsilon} 
\def \ch {{\cal H}} 
\def \del {\partial} 
\def \be {\begin{equation}} 
\def \ee {\end{equation}} 
\def \beq {\begin{eqnarray}} 
\def \eeq {\end{eqnarray}} 
\def \tv {\tilde v} 
\def \veren {\varepsilon^{\rm ren}_f} 
\def \vef {\varepsilon^0_f} 
\def \su {\uparrow} 
\def \sd {\downarrow} 
\def \CR {\nonumber\\} 
\def \hfb {\hfill\break} 
\def \tb {\bar{t} } 
\def \kb {\bar{k} } 
\def \tbB {\bar{t}_B } 
%\def \ul{#1} {$\underline{#1 }$}

\begin{document} 
\hspace{1cm}{\bf Title:} Observation of correlated x-ray scattering at atomic resolution \hfb
 
{\bf Authors:} Derek Mendez, Thomas J. Lane, Jongmin Sung,  Cl\`ement Levard, Herschel Watkins, Aina Cohen, Michael Soltis, Shirley Chisholm, James Spudich, Vijay Pande,  Daniel Ratner, Sebastian Doniach

{\bf Abstract}

Tools to study structure of disordered systems, such as proteins in solution, remain limited. Such understanding is essential for \emph{e.g.}~rational drug design. Correlated x-ray scattering (CXS) has recently attracted new interest as a way to leverage fourth generation light sources to study such disordered matter. The CXS experiment measures the intensity correlation caused by the scattering of x-rays from an ensemble of identical particles, with disordered orientation and position. Averaging over 15,496 images obtained by exposing a sample of silver nanoparticles in solution to a micro-focused synchrotron beam, we report for the first time an experimental CXS signal obtained form an ensemble in three dimensions. The correlation function was measured at  wide angles corresponding to atomic resolution and matches theoretical predictions. These results suggest that other CXS experiments on disordered ensembles -- such as proteins in solution -- may be possible in the future.


\section{Introduction}

In a pioneering paper, Kam \cite{Kam:1977wc} showed that correlated x-ray scattering (CXS) from an ensemble of randomly oriented particles could in principle reveal information about the internal structure of the particles beyond usual small and wide angle solution scattering measurements. The extraction of such information in the absence of an ordered system (\textit{e.g.}~a crystal) can be beneficial in biological studies, as many biological systems are inherently disordered (\textit{e.g.}~proteins in solution).

In order to gauge the feasibility of Kam's method to yield information about atomic length scales and to assess the difficulties associated with such a measurement, we conducted experiments measuring CXS from silver nanoparticle (NP) solutions at wide angles. Crucially, each measurement was conducted on an ensemble of NPs oriented randomly in three dimensions, extending previous work done in two dimensions \cite{Saldin:2011ch}, at small angles in 3-dimensions \cite{Kam:1981ua, Wochner:2009ia}, and on single particles in 3-dimensions \cite{Kam:1985tz, Starodub:1fy}. 

From these experiments, we obtained empirical correlation functions measuring the correlation between all pixel pairs in the first two silver powder rings. The three correlation functions (two rings with themselves, one between rings) show sharp peaks consistent with analytical and simulated predictions based on the FCC structure of silver. These peaks were deemed statistically significant by a Hotelling T-test [CITE] to a p-value of XXX [NEEEEEED THIS!].

By successfully measuring CXS signal from an ensemble of silver NPs, we have demonstrated the effectiveness of Kam's method given the current advances in x-ray technology. This experiment will serve as a benchmark for future experiments involving much weaker scatterers, such as proteins. It is our hope that the refinement of our our analysis techniques on a well known sample such as silver NPs will help facilitate the extension of CXS to studies of biomolecules in solution.

\section{Theory}

We briefly review the portions of \cite{Kam:1977wc} relevant to this manuscript. Let $S( \vec{q},\omega)$ represent the structure factor of an isolated particle in solution, i.e.

\be \label{structurefactor}
S(\vec{q},\omega) = \left| \> \int \rho \left(\hat{R} (\omega)\cdot \vec{r} \right) e^ { i \vec{q} \cdot \vec{r} } d\vec{r} \> \right| ^{2}
\ee

where $\rho $ is the electronic density throughout the particle volume, $\omega$ is a triple of Euler angles, and $\hat{R}(\omega)$ is a 3-dimensional rotation operator.

Kam showed that the correlation function

\be \label{correlation}
C(\vec{q}_1, \vec{q}_2) = \int S( \vec{q}_{1},\omega ) S( \vec{q}_{2},\omega ) \, d \omega %\,\, \equiv\,\, C(|\vec{q}_1|, |\vec{q}_2|,\psi)
\ee

may be extracted from a CXS measurement where one repeatedly records snapshots of a solution of $N$ identical particles, with each snapshot representing a unique ensemble of particles frozen in solution. Kam argued (assuming negligible inter-particle scattering interference) that averaging correlations of intensity fluctuations on each snapshot would result in the construction of (\ref{correlation}), i.e.

\be \label{converge}
\langle \delta n_{s}(\vec{q}_1) \, \delta n_{s}(\vec{q}_2) \rangle_{s} \Rightarrow C(\vec{q}_1, \vec{q}_2) 
\ee

where $n_{s}(\vec{q})$ is the total photons scattered from all particles in snapshot $s \,\,(1 \leq s \leq N_{s} )$ into a pixel along momentum transfer vector $\vec{q}$ and $\delta n_{s}(\vec{q}) = n_{s}(\vec{q}) - \langle n_{s}(\vec{q}) \rangle_{s}$. Neglecting inter-particle scattering interference, $n_{s}(\vec{q})$ can be thought of as a linear combination of $S(\vec{q},\omega)$ for a set of orientations $\{ \omega\}_{s}$.

It is crucial to emphasize that (\ref{correlation}) can be calculated using a model of $\rho$. Such a model can be refined against a CXS dataset using (\ref{converge}).

Silver NPs may be represented by a simple model consisting of a face-centered-cubic lattice cutoff by a spherical boundary. The scattering can then be represented by reciprocal lattice vectors cutting the Ewald sphere and giving rise to Bragg peaks broadened owing to the the finite size of the NPs.  Hence, each snapshot records a series of  Bragg rings. The sub-population of all NP orientations that simultaneously subtend two Bragg peaks on the Ewald sphere give rise to the correlations in (\ref{converge}). Let $\Delta$ be the angle between two scattering vectors $\vec{q}_{hkl}$ and $\vec{q}_{h'k'l'}$ satisfying the Bragg condition. Let $\phi$ be the azimuthal coordinate on the area detector recording each snapshot ( assumed here perpendicular to the forward scattering direction ) and define

\be \label{angular}
C (|\vec{q}_{hkl}|,|\vec{q}_{h'k'l'}|, \Delta  ) = \left \langle \int_{0}^{2\pi} \delta n_{s} (| \vec{q}_{hkl}|,\phi ) \,\, \delta n_{s} (|\vec{q}_{h'k'l'}|,\phi + \Delta )\, d\phi  \right \rangle_{s}
\ee

as the average angular correlation of Bragg ring intensity fluctuations. One should expect CXS signal then at values of $\Delta $ corresponding to the geometry of reciprocal lattice and the angles between reciprocal lattice vectors. [ NOT SURE WHETHER THIS IS HELPFUL OR HARMFUL For instance, $C (|\vec{q}_{111}|,|\vec{q}_{111}|, \Delta  )$ should display strong signal at $\Delta_1 = \arccos[ \frac{-2}{3\cos^{2}\theta} + 1  ]$ and $\Delta_2 = \arccos[ \frac{-4}{3\cos^{2}\theta} + 1  ]$ where $2\theta$ is the standard scattering angle.] 

Statistically, the CXS signal to noise scales as $\sqrt{N_{s}}$, but is independent of $N$.

\section{Methods}
To successfully measure a correlation function via the scheme (\ref{converge}), the sample must be frozen in time or space. Any random motion due to diffusion of particles will reduce the scattering correlation, which is a function of the particle structure and orientation (eq.~\ref{structurefactor}). To prevent diffusion during the long exposure times (order 1 second) necessary to scatter a sufficient number of photons to measure a correlation signal, we cooled the sample 100 Kelvin using a nitrogen cryo jet to ensure that the particles remained immobilized during each exposure. We had an estimated $10^{9}$  20 nm NPs per snapshot, but we observed significant numbers of NPs that were up to 50 nm in size.  The NPs were held in colloidal suspension with glycerol-based antifreeze used to prevent the formation of solvent crystals at the low temperature. By monitoring the intensity at constant scattering angle one can check for sample damage and diffusion.

To house the solutions we used kapton capillaries with a 500 and 600 $\mu$m inner and outer diameter respectively. Kapton scatters into relatively lower angles as does glycerol, hence we did not anticipate corrupting our silver nanoparticle signal with scattering from the kapton or glycerol.  The experiment was conducted at the micro-crystallography beamline (12-2) at
SSRL. Samples were prepared a day early and stored in a liquid nitrogen bath.

Samples were loaded and oriented in the X-ray beam using the Stanford Automated Mounting System (SAM), controllable from the experimental hutch. Using a liquid nitrogen-cooled double crystal monochromator we tuned the beam energy to 17 keV. The beam was focused down to about $20 \times 50 \mu$m$^2$ using Rh coated Kirkpatrick-Baez mirrors. Snapshots were recorded on a Dectris Pilatus 6M pixel detector. Our goal was to record as many snapshots as possible, each one representing a different ensemble of particle orientations frozen in time. The sample holder was equipped to automatically rotate the capillary 150 degrees about its longitudinal axis, perpendicular to the beam (Figure needed). Photon counts were read out and reset every 0.7 second as the capillary rotated 0.3 degrees under continuous beam irradiation, yielding 500 shots per 150 degree rotational scan. This was deemed an optimal timing to simultaneously maximize signal and minimize damage and heating. Every 500 shots, between scans, the capillary was moved longitudinally so as to always probe different regions of the sample.

Data from about 15,496 shots was collected and analyzed. A bicubic interpolation algorithm was used to convert the cartesian pixel lattice to polar coordinates for calculation of equations like (\ref{converge}). Using the Scherrer equation [CITE!!] relating the width of a Bragg ring to the average NP size, we concluded that the majority of silver NPs in each snapshot were roughly 20 nm in diameter. Histograms of photon counts into the $q_{111}$ Bragg ring seem to indicate a rather large distribution of particle sizes.

\section{Results}

We computed (\ref{angular} ) auto-correlations for $q_{111}$ and $q_{200}$. Initial attempts to see convergence of the CXS signal were unsuccessful. In addition to the statistical noises described in [CITE kirian, kam], there are several sources of systematic noise present in the experiment. The electronic response of each module seemed to fluctuate for reasons unknown, giving rise to dominant correlations between different detector regions. The polarization properties of the beam and sparse shadows on the detector also contributed to false correlations. All together, these systematic noises were enough to dominate the CXS which is already fighting against a strong statistical noise.

In order to eliminate any source of systematic noise in the intensity measurement we applied a binary filter to the data: intensities greater than a chosen threshold were set to unity [HOW DID YOU PICK THRESHOLD? WHY IS THAT BEST THRESHOLD?], the rest were set to zero (Figure 1).

[TJL stopped here, but below we should describe the data analysis]

%\begin{figure}[h]
%\begin{center}
%\includegraphics[height=7cm]{./fig3.pdf}
%\end{center}
%{\bf Figure 1:} Plot of the average of  $>$ 15,000 auto correlators computed from the first Bragg ring (dark line).  Shading around the line  is the standard error, vertical lines denote the correlator peaks from the double Bragg theory. The lower curve is the result of a  simulation of scattering from 1000 randomly oriented 20 nm nanoparticles.
%\end{figure}

[TJL thinks the below 3 paragraphs should just be cut. Instead, we should talk about (1) the difficulties in measuring CXS and how we overcame them. Then talk about *the data*. Discussions of particle size and photon counts go here, but should not be more than a para each.]

The signal/noise in these measurements results both from the intrinsic UDS noise and from the Poisson statistics of photon scattering. To gauge the relative magnitudes of these sources of noise we took advantage of the photon counting capability of the Pilatus detector. 

The total elastic scattering integrated over all angles is $n_{\rm scattered photons}\simeq\Phi\ N \sigma_{\rm nanoparticle}$ (up to a constant of $\cal O$(1)). Here $N$ is the number of particles in the beam $\Phi$ is the x-ray fluence and $\sigma_{\rm NP}$ is the coherent part of the x-ray scattering cross section for a nanoparticle (calculated from the atomic cross section taken from tables.) Integrating the photon counts over all 5 Bragg rings which are accessed by the 17keV x-rays, we get a total scattered photon count of order $10^9$ photons per shot. The x-ray fluence was $2\ 10^{12}$ photons into a focal spot of $20 \times 50 \mu$m$^2$ for a 0.5 second exposure. From the tabulated coherent cross section, and putting in 260,000 silver atoms/NP we estimate a scattering rate of 1.6 photons per NP per shot. We conclude that there were of order 6.3\ $10^8$ NPs in the beam.

From the estimates of the fraction of NPs giving CDS events given above, we conclude that the observed correlated scattering results from $\approx 0.1\%$ of the total scattering, giving an order of magnitude estimate of  $10^6$ CDS events per shot, thus showing that Poisson noise is not limiting the determination of the correlator profile in our simple case. Since our experiments show that averaging over a few thousand shots is adequate to separate the CDS events from the UDS background we conclude that the main impediment to accurate measurements of the correlated scattering comes from systemic errors resulting from  anisotropy artifacts induced by  the detector system, which we have been able to partially overcome by the nonlinear filtering to a binary signal. 

\section{Discussion}

[TJL next para should go]

As originally shown by Kam, the correlator for scattering from an arbitrary molecule averaged over random orientations, $C(\vec{q}_1,\vec{q}_2,\psi)$, with $\psi$ the angle between the scattering vectors, may be expanded in a series of Legendre polynomials  in $\psi$ whose coefficients may be directly calculated from the scattering structure factors of the molecule $F(\vec {q})=\sum_i f_i(q) \exp(i\vec {q}\cdot\vec {r}_i)$ where $\vec {r}_i$ are the atomic positions and $f_i(q)$ the atomic x-ray form factors. Hence accurate measurement of $C$ in the 3-dimensional $\{\vec{q}_1,\vec{q}_2,\psi\}$ space can lead to constraints which can be placed on the atomic positions, thus giving a route to iterative refinement of a given model (ref Brunger). 

[more like this next one]

Our results show that it is possible to obtain atomic scale information on the internal structure, for the very simple example  of a silver nanoparticle, for a bulk sample containing of order $10^8$ identical but randomly oriented particles. These results suggest that it should be feasible to obtain more detailed atomic scale constraints on models  of more complex biomolecules by measuring scattering using x-ray pulses from xFELs (refs Hajdu, Chapman - Spence). Such measurements have the potential to scatter many more photons/molecule, yielding more detailed $q_1,q_2,\psi$ information on the correlators. 

\section{D-References}

[Derek: check out my refs below, and make sure they match yours]

[1] Kam Z 1977 Determination of macromolecular structure in solution by spatial correlation of scattering fluctuations. Macromolecules 10 927-34

[2] Saldin D K, Poon H C, Bogan M J, Marchesini S, Shapiro D A, Kirian R A, Weierstall U and Spence J C H 2011 New light on disordered ensembles: ab initio structure determination of one particle from scattering fluctuations of many copies, Phys. Rev. Lett. 106 115501

[3] Kam Z, Koch M H J, and Bordas J 1981 Fluctuation x-ray-scattering from biological particles in frozen solution by using synchrotron radiation. Proc. Natl. Acad. Sci. USA 78 3559-62

[4] Wochner P, Gutt C, Autenrieth T, Demmer T, Bugaev V, Ortiz A D, Duri A, Zontone F, Grubel G and Dosch H 2009 X-ray cross correlation analysis uncovers hidden local symmetries in disordered matter. Proc. Natl Acad. Sci. USA 106 11511-4

[5] Kam Z and Gafni I 1985 3-dimensional reconstruction of the shape of human wart virus using spatial correlations. Ultramicroscopy 17 251�62

[6] Starodub D \textit{et al.} 2010 Single-particle structure determination by correlations of snapshot x-ray diffraction patterns. Nature Communications 3. (DOI: 10.1038/ncomms2288)


% TJL to Dermen:
% You gotta learn to use bibtex/linked references. You're life will be *much* better in the long run!
% my citations look crazy b/c they are auto-generated

\bibliographystyle{plain}
\begin{thebibliography}{1}

\bibitem{Kam:1977wc}
Z~Kam.
\newblock {Determination of macromolecular structure in solution by spatial
  correlation of scattering fluctuations}.
\newblock {\em Macromolecules}, 10(5):927--934, 1977.

\bibitem{Kam:1985tz}
Z~Kam and I~Gafni.
\newblock {Three-dimensional reconstruction of the shape of human wart virus
  using spatial correlations.}
\newblock {\em Ultramicroscopy}, 17(3):251--262, 1985.

\bibitem{Kam:1981ua}
Z~Kam, M~H Koch, and J~Bordas.
\newblock {Fluctuation x-ray scattering from biological particles in frozen
  solution by using synchrotron radiation.}
\newblock {\em Proceedings of the National Academy of Sciences},
  78(6):3559--3562, June 1981.

\bibitem{Saldin:2011ch}
D~Saldin, H~Poon, M~Bogan, S~Marchesini, D~Shapiro, R~Kirian, U~Weierstall, and
  J~Spence.
\newblock {New Light on Disordered Ensembles: Ab Initio Structure Determination
  of One Particle from Scattering Fluctuations of Many Copies}.
\newblock {\em Phys. Rev. Lett.}, 106(11):115501, March 2011.

\bibitem{Starodub:1fy}
D~Starodub, A~Aquila, S~Bajt, M~Barthelmess, A~Barty, C~Bostedt, J~D Bozek,
  N~Coppola, R~B Doak, S~W Epp, B~Erk, L~Foucar, L~Gumprecht, C~Y Hampton,
  A~Hartmann, R~Hartmann, P~Holl, S~Kassemeyer, N~Kimmel, H~Laksmono, M~Liang,
  N~D Loh, L~Lomb, A~V Martin, K~Nass, C~Reich, D~Rolles, B~Rudek, A~Rudenko,
  J~Schulz, R~L Shoeman, R~G Sierra, H~Soltau, J~Steinbrener, F~Stellato,
  S~Stern, G~Weidenspointner, M~Frank, J~Ullrich, L~Str~uuml der,
  I~Schlichting, H~N Chapman, J~C~H Spence, and M~J Bogan.
\newblock {Single-particle structure determination by correlations of snapshot
  X-ray diffraction patterns}.
\newblock {\em Nature Communications}, 3:1276--7, 1.

\bibitem{Wochner:2009ia}
Peter Wochner, Christian Gutt, Tina Autenrieth, Thomas Demmer, Volodymyr
  Bugaev, Alejandro~D{\'\i}az Ortiz, Agn{\`e}s Duri, Federico Zontone, Gerhard
  Gr{\"u}bel, and Helmut Dosch.
\newblock {X-ray cross correlation analysis uncovers hidden local symmetries in
  disordered matter.}
\newblock {\em P Natl Acad Sci Usa}, 106(28):11511--11514, July 2009.

\end{thebibliography}



\end{document}






